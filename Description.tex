 \documentclass[12pt]{article}
\usepackage[T2A]{fontenc}
\usepackage[utf8]{inputenc}       

\usepackage[russian, english]{babel}
\usepackage{amsmath,amsfonts,amsthm,amssymb,amsbsy,amstext,amscd,amsxtra,multicol}
\usepackage{verbatim}
\usepackage{tikz}
\usetikzlibrary{automata,positioning}
\usepackage{multicol}
\usepackage{graphicx}
\usepackage[colorlinks,urlcolor=blue]{hyperref}
\usepackage[stable]{footmisc}
\usepackage{ dsfont }
\usepackage{wrapfig}
\usepackage{xparse}
\usepackage{ifthen}
\usepackage{bm}
\usepackage{color}
 \usepackage{subfigure}
 
\usepackage{algorithm}
\usepackage{algpseudocode}

\usepackage{xcolor}
\usepackage{hyperref}
\definecolor{linkcolor}{HTML}{799B03} % цвет гиперссылок
\definecolor{urlcolor}{HTML}{799B03} % цвет гиперссылок
 
%\hypersetup{pdfstartview=FitH,  linkcolor=linkcolor,urlcolor=urlcolor, colorlinks=true}

\newtheorem{theorem}{Теорема}[section]
\newtheorem{lemma}{Лемма}[section]

\title{Когнитивный ассистент}

\DeclareMathOperator{\sign}{sign}
\DeclareMathOperator{\grad}{grad}
\DeclareMathOperator{\intt}{int}
\DeclareMathOperator{\conv}{conv}
\begin{document}

\maketitle

\section{Описание задачи}

В данной работе мы описываем методику конструирования целеориентированного режима когнитивного ассистента (далее КА). Нас интересует построение интерактивной диалоговой системы, т.е. системы, способной анализировать запросы пользователя, задавать ему вопросы и правильно на них отвечать, которая помогает пользователю в построении его собственного плана достижения заранее (в самом начале диалога) обозначенной цели. А именно, цель пользователя является достаточно конкретной (напр., купить автомобиль), и КА должен в процессе диалога максимально конкретизировать цель (в случае автомобиля данной конкретезацией может быть выбор между отечественным автомобилем и иномаркой, между подержанным и новым и т.д.) и провести пользователя по всем этапам достижения цели давая рекомендации.

Мотивацией для создания новой модели поведения когнитивного ассистента является следующее. На данный момент диалоговые системы не показывают результатов, сравнимых с результатом общения с экспертом. Во-первых, текущие технологии не заточены на интерактивное общение с пользователем. А именно, существуют неплохие вопросно ответные системы \cite{1} и есть результаты в создании диалоговых систем для общения на свободную тему (chit-chat) \cite{2}, однако практически не разработаны системы, задающие вопрос пользователю и обрабатывающие ответ для достижения собственных целей. Во-вторых, существующие диалоговые системы плохо заточены для работы в предметной области. Для решения этих задвч в частности используются базы знаний \cite{3}.

\section{Модель ассистента}

В данной работе мы предполагаем, что у когнитивного ассистента есть некоторый заранее прописаный экспертом сценарий поведения. Обсудим структуру данных, которую мы будем использовать для работы со сценарием.

Мы будем представлять сценарий нашего ассистента как ориентированный граф. Перед формализацией этого графа учтем, что для общения с пользователем и помощи ему в достижении цели нам необходимо знать два существенных параметра: текущий этап планирования и параметры пользователя, т.е. его интересы и индивидуальные особенности. Этап планирования будет определяться по текущей вершине в графе. Будем считать, что состоянию пользователя соответствует некоторый численный вектор $\textbf{x}\in \mathbb{R}^n$.

Тогда графом сценария будет следующий ориентированный граф:

\begin{equation}
\begin{aligned}
&G = (V,E, \phi) \\
&s, t \in V \\
&E \subset V\backslash\{t\}\times V\\
&\phi:E\times \mathbb{R}^n \rightarrow \mathbb{R}_+
\end{aligned}
\end{equation}
где вершины $s,t$ - вершины начала диалога и окончания. Структуру элементов множества вершин мы обсудим ниже. В данной модели любое ребро $(a,b)$ определяет существующую возможность перехода из этапа соответствующего вершине $a$ к этапу, соответствующему вершине $b$. Заметим, что в определении множества ребер мы поставили естественное условие на отсутствие ребер с началом в терминальной вершине $t$. Функция $\phi$ определяет то, насколько переход по текущему ребру соответсвует интересам и особенностям пользователя.

Теперь обсудим структуру вершины. Каждая вершина соответствует некоторому этапу планирования достижения цели пользователя, поэтому, когда ассистент достигает новой вершины, пользователь должен получить некоторую полезную информацию. Из этого следует, что каждой вершине соответствует процедура генерации текста.

В предположении если нам известен вектор $\textbf{x}$ пользователя, то когнитивный ассистент может построить путь от $s$ до $t$, выбирая в каждой вершине ребро по максимуму $\phi$, и пройти путь, выводя рекомендации для пользователя. Определить вектор $\textbf{x}$ можно заранее опросив пользователя по всем пунктам. Однако данный подход имеет следующие недостатки. Во-первых, это достаточно неудобно с точки зрения пользователя. Так для того, чтобы он смог получить некоторую полезную информацию, он должен провести достаточно долгую подготовительную работу - ответить на большой перечень вопросов. Во-вторых, данный метод не учитывает того, что интересы пользователя могут меняться во время диалога. Тогда, чтобы обновить $\textbf{x}$, нам нужно провести опять опрос. В-третьих, эта система не учитывает того, что уже на стадии опроса нужно уметь давать рекомендации пользователю, чтобы он смог определиться.

Для устранения этих недостатков мы будем использовать иную схему моделирования картины мира пользователя. Заметим, что нас будет интересовать $\textbf{x}$ только в таких вершинах $u$, что $|E \cap (\{u\}\times V)|>2$, т.е. где есть выбор между возможными следующими этапами. Тогда пусть  $u$ - это такая вершина. Тогда определим множество смежных с $u$ вершин:
$$V_u = \left\{v|(u,v)\in E\right\}$$
и функцию выбора оптимального ребра:
$$\Phi_u(\textbf{x}) = \arg\max_{e\in\left\{(u,v)|(u,v)\in E\right\}}\phi(e,\textbf{x}).$$
Естественно ожидать, что если $n \gg 1$, то функция $\Phi_u$ скорей всего зависит только от малого количества компонент $\textbf{x}$, из чего следует, что в вершине $u$ нам достаточно определить только эти компоненты. Обозначим множество индексов этих компонент как $J_u$. Заметим, что часть из этих компонент может быть уже определена.

Тогда для каждой вершины $u$ определим создадим вспомогательную диалоговую систему, которая определяется следующими целями и возможностями:

1) Цель диалога определить множество $\{x_k\}_{k\in J_u}$

2) Ассистент может изменить любую компоненту $\textbf{x}$, если это следует из запросов пользователя.

Далее мы обсудим способ построения такой системы, но перед этим предполагается формализовать $\textbf{x}$. В рамках данной работы предполагается работать и хранить не сам $\textbf{x}$, а его во множество $\mathbb{R}^{|E|}_+$. Допустим, что мы пронумеровали наши ребра. Тогда необходимое отображение определяется следующим образом

$$\left[\Psi(\textbf{x})\right]_k = \phi(e_k, \textbf{x}),$$
и тогда новая функция соответствия $$\tilde{\phi}:E\times\mathbb{R}^{|E|}\rightarrow \mathbb{R}_+$$ будет определяться как
$$\tilde{\phi}(e_k, \textbf{x}) = x_k$$
Заметим, что данное отображение полностью удовлетворяет вышесказанному предположению о том, что $\Phi_u$ зависит от малого количества переменных. При данном определении это количество равно количеству смежных вершин.

\subsection{Функция $\phi$}

В силу того, что сценарий построен экспертом, логично предполагать, что для каждого ребра условие перехода будет задано текстом. Тогда признак $x_k$ есть соответветствие между пользователем тексту для ребра $e_k$.

\begin{thebibliography}{3}
\bibitem{1}
???
\bibitem{2}
???
\bibitem{3}
???
\end{thebibliography}

\end{document}
